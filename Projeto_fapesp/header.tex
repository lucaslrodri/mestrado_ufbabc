\documentclass[
	openboth, % Capítulos começam em página direita
	oneside,   % Impressão frente e verso
	a4paper,   % Tamanho do papel
	english,   % Idioma adiciona para hifenizaçãoo
	brazil     % O ultimo é o idioma principal
]{abntex2}

\usepackage{lmodern}                               % Usa família de fontes Latin Modern
\usepackage[T1]{fontenc}                           % Codifição Type1 para a fonte (Fontes de 8 bits para incluir as fontes do alfabeto brasileiro)
\usepackage[UTF8]{inputenc}                        % Codificação unicode "UTF8"
\usepackage{microtype}                             % Melhorias na justificação (Evita bad/underfullbox)
\usepackage{indentfirst}                           % Indenta o primeiro parágrafo da seção
\usepackage{amssymb}                               % Símbolos matemáticos
\usepackage{enumitem}                              % Melhoria no suporte aos ambientes de numeração
\usepackage{color}                                 % Suporte a cores
\definecolor{blue}{RGB}{41,5,195}                  % Alterando o aspécto da cor Azul
\usepackage{graphicx}                              % Suporte a gráficos e figuras
\usepackage[alf]{abntex2cite}                      % Citações no padrão ABNT
\usepackage{mathtools}                             % Equações
\usepackage{textcomp}                              % Símbolos
\usepackage{makecell}                              % Melhorias em tabelas
\usepackage{array}                                 % Melhorias em tabelas

%Novos comandos

\renewcommand{\imprimircapa}[2]{
	\begin{capa}%
		\centering
		\begin{center}
			\ABNTEXchapterfont\large\textbf{\MakeUppercase\imprimirinstituicao}
		\end{center}
		\vfill
		\begin{center}
			\ABNTEXchapterfont\bfseries\LARGE\MakeUppercase\imprimirtitulo
		\end{center}
		\vspace{2cm}
		\begin{center}
			\large\imprimirpreambulo
		\end{center}
		\vspace*{2cm}
		\begin{center}
			\large
			\begin{tabular}{rl}
				Orientador: & \imprimirorientador \\
				Email: & #1 \\
				Beneficiário: & \imprimirautor \\
				Email: & #2
			\end{tabular}
		\end{center}
		\vspace*{\fill}
		{\large\MakeUppercase\imprimirlocal} \\
		{\large\MakeUppercase\imprimirdata}
		\vspace*{1cm}
	\end{capa}
}



% informações do PDF
\makeatletter
\hypersetup{
	%pagebackref=true,
	pdftitle={\@title}, 
	pdfauthor={\@author},
	pdfsubject={\imprimirpreambulo},
	pdfcreator={LaTeX with abnTeX2},
	pdfkeywords={abnt}{latex}{abntex}{abntex2}{trabalho acadêmico}, 
	colorlinks=true,       		% false: boxed links; true: colored links
	linkcolor=black,          	% color of internal links
	citecolor=black,        		% color of links to bibliography
	filecolor=black,      		% color of file links
	urlcolor=black,
	bookmarksdepth=4
}
\makeatother

% O tamanho do parágrafo é dado por:
\setlength{\parindent}{1.3cm}

% Controle do espaçamento entre um parágrafo e outro:
\setlength{\parskip}{0.2cm}  % tente também \onelineskip


